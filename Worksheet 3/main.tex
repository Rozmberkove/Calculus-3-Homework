\documentclass[letter,11pt]{article}
\usepackage{latexsym}
\usepackage{xcolor}
\usepackage{float}
\usepackage{amsthm}
\usepackage{amssymb}
\usepackage{wrapfig}
\usepackage{tabularx}
\usepackage{titlesec}
\usepackage{tikz}
\usepackage{geometry}
\usepackage{verbatim}
\usepackage{enumitem}
\usepackage{fancyhdr}
\usepackage{pgfornament}
\usepackage{multicol}
\usepackage{graphicx}
%\usepackage{cfr-lm}
\usepackage{booktabs}
\usepackage{svg}
\usepackage[T1]{fontenc}
\usetikzlibrary{trees}
\setlength{\multicolsep}{0pt} 
\pagestyle{fancy}
%\fancyhf{} % clear all header and footer fields
\fancyhead{}\fancyfoot{}
\fancyhead[R]{\textbf{\thepage}}
\fancyhead[L]{Aiden M. Rosenberg, MMXXIII A.D. }
\addtolength{\headwidth}{3cm}


\renewcommand{\headrulewidth}{1pt}
\renewcommand{\footrulewidth}{0pt}
\geometry{left=1.5cm, top=2.5cm, right=1.5cm, bottom=2cm}

%\usepackage{draftwatermark}	
%\SetWatermarkColor[gray]{0.9}
%\SetWatermarkText{Private}
%\SetWatermarkScale{3}

\usepackage[most]{tcolorbox}
\tcbset{
	frame code={}
	center title,
	left=0pt,
	right=0pt,
	top=0pt,
	bottom=0pt,
	colback=gray!20,
	colframe=white,
	width=\dimexpr\textwidth\relax,
	enlarge left by=-2mm,
	boxsep=4pt,
	arc=0pt,outer arc=0pt,
}


\raggedright
\setlength{\tabcolsep}{0in}

% Sections formatting
\titleformat{\section}{
  \vspace{-4pt}\scshape\raggedright\large
}{}{0em}{}[\color{black}\titlerule \vspace{-7pt}]

\begin{document}

\thispagestyle{empty}

\fontfamily{cmr}\selectfont
%----------HEADING-----------------

\parbox{2.35cm}{%
  \includesvg[width=2.3cm]{logo.svg}
}
\parbox{0.3cm}{\hspace{0.3cm}}
\parbox{\dimexpr\linewidth-5cm\relax}{
\setlength{\tabcolsep}{0.5em}
\def\arraystretch{1.25}
\begin{tabular}{@{}llll@{}}
\toprule
 \multicolumn{4}{c}
 {\hspace{-0.5em}\textbf{Assignment}: Worksheet \#3 (15.2-15.4)} \\ \midrule
\textbf{Name:}      & D. Aiden M. Rosenberg    & \textbf{Professor:}   & Dr. Alan v. Herrmann Ph.D        \\
\textbf{Course:}    & Calculus III        & \textbf{Date:}        & September 7th, 2023 A.D.   \\ \bottomrule
\end{tabular}
}
\vspace{1cm}

\section*{Section 15.2}
 For each of the following, find the limit or prove that is does not exist:
\begin{enumerate}[label = \Alph*.]
    \item $\lim_{(x,y)\to (1,1)} \frac{x^2-2xy+y^2}{x-y}$
        \begin{align*}
            \lim_{(x,y)\to (1,1)} \frac{x^2-2xy+y^2}{x-y} &= \lim_{(x,y)\to (1,1)} \frac{(x - y)^2}{x - y} \\
            &= \lim_{(x,y)\to (1,1)} (x - y) \\
            &= 1 - 1 \\
            &= 0.
        \end{align*}
    \item $\lim_{(x,y)\to (0,0)} \frac{(x+y)^2}{x^2+y^2}$
    
    \vspace{0.5cm}
    \textit{Method I:}
        \begin{enumerate}
            \item Let $x=y \Longrightarrow \lim_{(x,y)\to (0,0)} \frac{(y+y)^2}{y^2+y^2} = \lim_{(x,y)\to (0,0)} \frac{4y^2}{2y^2} = 2$
            \item Let  $x=y^2 \Longrightarrow \lim_{(x,y)\to (0,0)} \frac{(y^2+y)^2}{y^4+y^2} = \lim_{(x,y)\to (0,0)} \frac{y^2+2y+1}{y^4+y^2}  = \lim_{(x,y)\to (0,0)} \frac{y^4+2y^3+y^2}{y^2+1} = 1$
        \end{enumerate}
        Since different paths lead to different limiting values, the given limit does not exist.
    
    \vspace{0.5cm}
    \textit{Method II:}
        \begin{align*}
            \lim_{(x,y)\to (0,0)} \frac{(x+y)^2}{x^2+y^2} &= \lim_{r\to 0} \frac{(r\cos\theta + r\sin\theta)^2}{(r\cos\theta)^2 + (r\sin\theta)^2} \\
            &= \lim_{r\to 0} \frac{r^2(\cos\theta + \sin\theta)^2}{r^2(\cos^2\theta + \sin^2\theta)} \\
            &= \lim_{r\to 0} \frac{(\cos\theta + \sin\theta)^2}{\cos^2\theta + \sin^2\theta} \\
            &= \lim_{r\to 0} (\cos\theta + \sin\theta)^2.
        \end{align*}
    Now, consider the limit as $r$ approaches $0$. The value of $\theta$ can vary as $(x,y)$ approaches $(0,0)$, so we need to consider all possible values of $\theta$.
    \begin{enumerate}
        \item When $\theta = 0$ we have $(\cos\theta + \sin\theta)^2 = (1+0)^2=1$
        \item When $\theta = \frac{\pi}{4}$ we have $(\cos\theta + \sin\theta)^2 = (\frac{\sqrt{2}}{2}+\frac{\sqrt{2}}{2})^2=\frac{1}{4}$
    \end{enumerate}
    Since the limit is different for both cases, we can conclude that the limit of the given expression does not exist.
    \newpage
    \item $\lim_{(x,y)\to (0,0)} \frac{2x^2y}{2x^4+y^2}$
        \begin{enumerate}
            \item Let $y=x^2 \Longrightarrow \lim_{(x,y)\to (0,0)} \frac{2x^2 \cdot x^2}{2x^4+x^4} = \lim_{(x,y)\to (0,0)} \frac{2x^4}{3x^4}= \frac{2}{3}$
            \item Let $y=x \Longrightarrow \lim_{(x,y)\to (0,0)} \frac{2x^2 \cdot x}{2x^4+x^2} = \lim_{(x,y)\to (0,0)} \frac{2x^4}{2x^4+x^2}=\lim_{(x,y)\to (0,0)} \frac{2x}{2x^2+1} = 0$
            \item Since different paths lead to different limiting values, the given limit does not exist.
        \end{enumerate}
    \item $\lim_{(x,y)\to (0,0)} \frac{\sin \sqrt{2x^2+2y^2}}{\sqrt{x^2+y^2}}$

        \begin{align*}
\lim_{(x,y)\to (0,0)} \frac{\sin \sqrt{2x^2+2y^2}}{\sqrt{x^2+y^2}} &= \lim_{r\to 0} \frac{\sin \sqrt{2(r\cos\theta)^2+2(r\sin\theta)^2}}{\sqrt{(r\cos\theta)^2+(r\sin\theta)^2}} \\
&= \lim_{r\to 0} \frac{\sin \sqrt{2r^2(\cos^2\theta + \sin^2\theta)}}{r\sqrt{\cos^2\theta + \sin^2\theta}} \\
&= \lim_{r\to 0} \frac{\sin \sqrt{2r^2}}{r} \\
&= \lim_{r\to 0} \frac{\sin (\sqrt{2}r)}{r}.
\end{align*}
Now, we can evaluate the limit as $r$ approaches 0:
$$\sqrt{2} \lim_{r\to 0} \frac{\sin (\sqrt{2}r)}{\sqrt{2}r} = \sqrt{2} \cdot 1 = \sqrt{2}.$$
\end{enumerate}
\section*{Section 15.3}
Find the first partial derivatives of the following functions.
\begin{enumerate}
    \item $f(x,y)=xe^{4x^2-y}$
        \begin{enumerate}
            \item $F_x = \frac{\partial f}{\partial x} = e^{4x^2-y}+ 8x^2e^{4x^2-y} = e^{4x^2-y}(1+8x^2)$
            \item $F_y = \frac{\partial f}{\partial y} = -e^{4x^2-y}$
        \end{enumerate}
    \item $f(x,y,z) = \frac{2x-z}{\sqrt{x^2+y^2}}$
        \begin{enumerate}
            \item $F_x=  \frac{\partial f}{\partial x} = \frac{2\sqrt{x^2+y^2}-x(x^2+y^2)^{-1/2}(2x-z)}{x^2+y^2} = \frac{2}{\sqrt{x^2+y^2}} - \frac{x(2x-z)}{(x^2+y^2)^{3/2}} = \frac{zx+2y^2}{(x^2+y^2)^{3/2}}$
            \item $F_y=  \frac{\partial f}{\partial y} = \frac{2y(2x-z)}{(x^2+y^2)^{3/2}}$
            \item $F_z=  \frac{\partial f}{\partial z} = \frac{-1}{\sqrt{x^2+y^2}}$
        \end{enumerate}
\end{enumerate}
\section*{15.3 cont.}
Partial derivatives are used to model a variety of scenarios in a variety of applications. For example, the wave equation models the propagation of a wave as a function of space and time
$$\frac{\partial^2u}{\partial t^2} = c^2\frac{\partial^2u}{\partial x^2}.$$
Show that $u(x, t) = \tan(2x - 2ct)$ satisfies the wave equation.
\begin{enumerate}
    \item $\frac{\partial u}{\partial t} = -2c\sec^2(2x-2ct) \Longrightarrow \frac{\partial^2 u}{\partial t^2}= \frac{\partial u}{\partial t}\left(-2c\sec^2(2x-2ct)\right) = 8c^2\sec^2(2x-2ct)\cdot \tan(2x-2ct)$
    \item $\frac{\partial u}{\partial x} = 2\sec^2(2x-2ct) \Longrightarrow \frac{\partial^2 u}{\partial x^2}= \frac{\partial u}{\partial x}\left(2\sec^2(2x-2ct)\right) = 8\sec^2(2x-2ct)\cdot \tan(2x-2ct)$
\end{enumerate}
Thus, for $u(x,t)$, substituting the values found above $\frac{\partial^2u}{\partial t^2} = c^2\frac{\partial^2u}{\partial x^2}$ indeed satisfies the wave equality.
\section*{Section 15.4} 
Use implicit differentiation to find the indicated derivative for each of the given equations.

\begin{enumerate}[label=\Alph*.]
    \item Find $\frac{\partial z}{\partial x}$ for $\cos(2x + z^2) = 4x - 3z$. Assume that the equation implicitly defines $z$ as a differentiable function of $x$ and $y$.
        \begin{align*}
            \frac{d}{dx}\left[\cos(2x + z^2)\right] &= \frac{d}{dx}\left[4x - 3z\right]\\
             -\sin(2x + z^2) \cdot \frac{d}{dx}(2x + z^2) &= 4 - 3\frac{dz}{dx} \\
             -\sin(2x + z^2) \cdot (2 + 2z \frac{d}{dz}) &= 4 - 3\frac{dz}{dx} \\
             -2\sin(2x + z^2) - 2z\sin(2x + z^2)\frac{dz}{dx} &= 4 - 3\frac{dz}{dx} \\
             \left(-2z\sin(2x + z^2) + 3\right)\frac{dz}{dx} &= 4 + 2\sin(2x + z^2)\\
             \frac{dz}{dx} &= \frac{4 + 2\sin(2x + z^2)}{-2z\sin(2x + z^2) + 3} \\
             \frac{\partial z}{\partial x} &= \frac{4 + 2\sin(2x + z^2)}{-2z\sin(2x + z^2) + 3}
        \end{align*}
    \item Find $\frac{\partial z}{\partial y}$ for $3xy^2+z=\ln(4-z)$.Assume that the equation implicitly defines $z$ as a differentiable function of $x$ and $y$.
    \begin{align*}
        \frac{d}{dy}[3xy^2 + z] &= \frac{d}{dy}\left[\ln(4 - z)\right]\\
        3x\frac{d}{dy}(y^2) + \frac{dz}{dy} &= \frac{d}{dz}\left(\ln(4 - z)\right) \cdot \frac{d}{dy}(4 - z)\\
        3x(2y) + \frac{dz}{dy} &= \frac{1}{4 - z} \cdot \left(\frac{d}{dy}(4 - z)\right) \\
        6xy + \frac{dz}{dy} &= \frac{1}{4 - z} \cdot (-\frac{dz}{dy}) \\
        (4 - z)\frac{dz}{dy} &= -\frac{dz}{dy} - 6xy(4 - z) \\
        (4 - z + 1)\frac{dz}{dy} &= - 6xy(4 - z)\\
        \frac{dz}{dy} &= \frac{- 6xy(4 - z)}{5 - z}\\
         \frac{\partial z}{\partial y} &= \frac{- 6xy(4 - z)}{5 - z}
\end{align*}
\end{enumerate}
\newpage
\section{Section 15.4 cont.}
 Let $z = \ln(5x-y)$, where $x = u-6v$ and $y = 2u + 3v$. Draw the derivative tree diagram and then find $\frac{\partial z}{\partial v}$ using the chain rule. Write your answer in terms of $u$ and $v$.

\begin{figure}[ht]
\centering
\begin{tikzpicture}[level distance=2cm,
  level 1/.style={sibling distance=4cm},
  level 2/.style={sibling distance=3cm},
  level 3/.style={sibling distance=2cm},
  every node/.style={text width=3cm, align=center}]

  \node {$z = \ln(5x - y)$}
    child {node {$x = u - 6v$}
      child {node {$u$}}
      child {node {$v$}}}
    child {node {$y = 2u + 3v$}
      child {node {$u$}}
      child {node {$v$}}};

\end{tikzpicture}
\caption{Derivative Tree Diagram}
\end{figure}
\begin{align*}
\frac{\partial z}{\partial v} &= \frac{\partial z}{\partial x} \cdot \frac{\partial x}{\partial v} + \frac{\partial z}{\partial y} \cdot \frac{\partial y}{\partial v}\\
&= \left(\frac{5}{5x-y} \cdot -6\right)+\left(\frac{-1}{5x-y} \cdot 3\right)\\
&= \frac{-30}{5x-y} - \frac{3}{5x-y} \\
&= \frac{-33}{5x-y}\\
&= \frac{-33}{5(u-6v)-(2u + 3v)}\\
&= \frac{-33}{3u-33v}
\end{align*}
\end{document}
