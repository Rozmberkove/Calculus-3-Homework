\documentclass[letter,11pt]{article}
\usepackage{latexsym}
\usepackage{xcolor}
\usepackage{float}
\usepackage{amsthm}
\usepackage{esint}
\usepackage{amssymb}
\usepackage{wrapfig}
\usepackage{tabularx}
\usepackage{empheq}
\usepackage{titlesec}
\usepackage{tikz}
\usepackage{geometry}
\usepackage{verbatim}
\usepackage{epstopdf}
\usepackage{enumitem}
\usepackage{fancyhdr}
\usepackage{pgfornament}
\usepackage{multicol}
\usepackage{systeme}
\usepackage{graphicx}
\usepackage{mathtools}
%\usepackage{cfr-lm}
\usepackage{booktabs}
\usepackage{svg}
\usepackage[T1]{fontenc}
\usetikzlibrary{trees}
\setlength{\multicolsep}{0pt} 
\pagestyle{fancy}
%\fancyhf{} % clear all header and footer fields
\fancyhead{}\fancyfoot{}
\fancyhead[R]{\textbf{\thepage}}
\fancyhead[L]{Aiden M. Rosenberg, MMXXIII A.D. }
\addtolength{\headwidth}{3cm}

\usepackage{pgfplots}
\pgfplotsset{compat=1.17}
\usepgfplotslibrary{fillbetween}

\usepackage{pst-plot}
\usepgfplotslibrary{polar}



\renewcommand{\headrulewidth}{1pt}
\renewcommand{\footrulewidth}{0pt}
\geometry{left=1.5cm, top=2.5cm, right=1.5cm, bottom=2cm}

%\usepackage{draftwatermark}	
%\SetWatermarkColor[gray]{0.9}
%\SetWatermarkText{Private}
%\SetWatermarkScale{3}

\usepackage[most]{tcolorbox}
\tcbset{
	frame code={}
	center title,
	left=0pt,
	right=0pt,
	top=0pt,
	bottom=0pt,
	colback=gray!20,
	colframe=white,
	width=\dimexpr\textwidth\relax,
	enlarge left by=-2mm,
	boxsep=4pt,
	arc=0pt,outer arc=0pt,
}


\raggedright
\setlength{\tabcolsep}{0in}

% Sections formatting
\titleformat{\section}{
  \vspace{-4pt}\scshape\raggedright\large
}{}{0em}{}[\color{black}\titlerule \vspace{-7pt}]

\titleformat{\subsection}[block]
  { \vspace{4pt}\bfseries\centering}
  {}{0em}{}

\begin{document}

\thispagestyle{empty}

\fontfamily{cmr}\selectfont
%----------HEADING-----------------

\parbox{2.35cm}{%
	\includesvg[width=2.3cm]{logo.svg}
}
\parbox{0.3cm}{\hspace{0.3cm}}
\parbox{\dimexpr\linewidth-5cm\relax}{
	\setlength{\tabcolsep}{0.5em}
	\def\arraystretch{1.25}
	\begin{tabular}{@{}llll@{}}
		\toprule
		\multicolumn{4}{c}
		{\hspace{-0.5em}\textbf{Assignment}: Worksheet \#9 (6.6-16.7)} \\ \midrule
		\textbf{Name:}   & D. Aiden M. Rosenberg & \textbf{Professor:} & Dr. Alan v. Herrmann Ph.D \\
		\textbf{Course:} & Calculus III          & \textbf{Date:}      & October 21st, 2023 A.D.   \\ \bottomrule
	\end{tabular} }
\vspace{1cm}

\section*{Section 16.6}
Find the center of mass of the solid given by $W$ bounded above by $z = r^2$, below by $z = 0$ and laterally by $r = 1$. with density function $\delta(r, \theta, z) = r$.
\begin{align*}
    M&= \int_{0}^{2\pi}\int_{0}^{1}\int_{0}^{r^{2}}\delta(r, \theta, z) r \, dz\, dr\, d\theta\\
    &= \int_{0}^{2\pi}\int_{0}^{1}\int_{0}^{r^{2}} r^2 \, dz\, dr\, d\theta\\
    &= \int_{0}^{2\pi}\int_{0}^{1} \left[r^2z\right]_{0}^{r^2}dr\, d\theta \\
    &= \int_{0}^{2\pi}\int_{0}^{1} r^4 dr\, d\theta \\
    &= \int_{0}^{2\pi}\left[\frac{r^4}{5}\right]_{0}^{1} dr\, d\theta \\
    &= \frac{1}{4}\int_{0}^{2\pi}d\theta \\
    &= \left[\frac{1}{4}\theta\right]_{0}^{2\pi} \\
    &= \frac{2\pi}{5}
\end{align*}
$x=r\cos\theta$, $y=r\sin\theta$ and $z=z$.
\begin{align*}
\bar{x} &= \frac{1}{M}\int_{0}^{2\pi}\int_{0}^{1}\int_{0}^{r^{2}} x\delta(r, \theta, z) r \, dz\, dr\, d\theta \\
        &= \frac{5}{2\pi}\int_{0}^{2\pi}\int_{0}^{1}\int_{0}^{r^{2}} r^3\cos\theta \, dz\, dr\, d\theta \\
        &= \frac{5}{2\pi}\int_{0}^{2\pi}\int_{0}^{1} \left[r^3\cos\theta z\right]_{0}^{r^{2}} \, dr\, d\theta \\
        &= \frac{5}{2\pi}\int_{0}^{2\pi}\int_{0}^{1} r^5\cos\theta \, dr\, d\theta \\
        &= \frac{5}{2\pi}\int_{0}^{2\pi} \left[\frac{r^6}{6}\cos\theta \right]_{0}^{1}\, d\theta \\
        &= \frac{5}{12\pi}\int_{0}^{2\pi} \left[\cos\theta \right]\, d\theta \\
        &= \left[\frac{5}{2\pi} \sin\theta \right]_{0}^{2\pi} \\
        &= 0 \\
\end{align*}
\begin{align*}
\bar{y} &= \frac{1}{M}\int_{0}^{2\pi}\int_{0}^{1}\int_{0}^{r^{2}} z\delta(r, \theta, z) r \, dz\, dr\, d\theta \\
        &= \frac{5}{2\pi}\int_{0}^{2\pi}\int_{0}^{1}\int_{0}^{r^{2}} r^3\sin\theta \, dz\, dr\, d\theta \\
        &= \frac{5}{2\pi}\int_{0}^{2\pi}\int_{0}^{1} \left[r^3\sin\theta z\right]_{0}^{r^{2}} \, dr\, d\theta \\
        &= \frac{5}{2\pi}\int_{0}^{2\pi}\int_{0}^{1} r^5\sin\theta \, dr\, d\theta \\
        &= \frac{5}{12\pi}\int_{0}^{2\pi} \left[\frac{r^6}{6}\sin\theta \right]_{0}^{1}\, d\theta \\
        &= \frac{5}{12\pi}\int_{0}^{2\pi} \left[\sin\theta \right]\, d\theta \\
        &= \left[-\frac{5}{12\pi}\cos\theta \right]_{0}^{2\pi} \\
        &= 0 
\end{align*}
\begin{align*}
\bar{z} &= \frac{1}{M}\int_{0}^{2\pi}\int_{0}^{1}\int_{0}^{r^{2}} z\delta(r, \theta, z) r \, dz\, dr\, d\theta \\
        &= \frac{5}{2\pi}\int_{0}^{2\pi}\int_{0}^{1}\int_{0}^{r^{2}} zr^2 \, dz\, dr\, d\theta \\
        &= \frac{5}{2\pi}\int_{0}^{2\pi}\int_{0}^{1} \left[\frac{z^2}{2}r^{2}\right]_{0}^{r^{2}} \, dr\, d\theta \\
        &= \frac{5}{2\pi}\int_{0}^{2\pi}\int_{0}^{1} \frac{r^6}{2} \, dr\, d\theta\\
        &= \frac{5}{2\pi}\int_{0}^{2\pi}\left[ \frac{r^7}{14} \right]_{0}^{1}\, \, d\theta\\
         &= \frac{5}{28\pi}\int_{0}^{2\pi} \, d\theta\\
         &= \left[\frac{5}{28\pi}\theta\right]_{0}^{2\pi} \, d\theta\\
         &=\frac{10\pi}{28\pi}\\
         &= \frac{5}{14}
\end{align*}

$$\left(\bar{x}, \bar{y}, \bar{z}\right)= \boxed{\left(0, 0, \frac{5}{14}\right)}$$
\section*{Section 16.6, 16.7}
Consider lamina (thin plate) $R$ is the region in the $xy-$plane bounded by $y = \frac{1}{4}x$, $y = \frac{5}{2}x$, and the hyperbolas $xy = 1$ and $xy = 5$. Use the transformation $u = \frac{y}{x}$ and $v = xy$ to transform $R$ in the $xy-$plane into region $S$ in the $uv-$plane.

\begin{enumerate}[label = \roman*.]
    \item Sketch the original region of integration $R$ in the $xy-$plane and the new region $S$ in the $uv-$plane using the given change of variables.
    \begin{figure}[h]
\begin{minipage}{0.5\textwidth}
    \centering
    \begin{tikzpicture}[scale=0.75]
        \begin{axis}[
            axis lines = middle,
            xlabel = $x$,
            ylabel = $y$,
            ymin = -2,
            ymax = 5,
            xmin = -2,
            xmax = 5,
        ]

        % Plot y = 1/4 * x
        \addplot[black,dashed, domain=2:4.472, samples=400, name path=A] {1/4*x};

        % Plot y = 5/2 * x
        \addplot[black,dashed, domain=0.632:1.414, samples=400, name path=B] {5/2*x};

        % Plot xy = 1
        \addplot[black,dashed, domain=0.632:2, samples=400,name path=C] {1/x};

        % Plot xy = 5
        \addplot[black,dashed, domain=1.414:4.472, samples=400,  name path=D] {5/x};

        \node at (2, 1.5) {$R$}; % Label R

        % Shade the area between the curves
        \addplot[blue!20 , opacity=0.5] fill between[of=A and C];
        \addplot[blue!20 , opacity=0.5] fill between[of=A and D];
        \addplot[blue!20 , opacity=0.5] fill between[of=B and C];
        \end{axis}
    \end{tikzpicture}
    \caption{Graph of the region $R$}
\end{minipage}%
\begin{minipage}{0.5\textwidth}
    \centering
    \begin{tikzpicture}[scale=0.75]
        % Axes
        \draw[->] (-0.5,0) -- (3,0) node[below] {$u$};
        \draw[->] (0,-0.5) -- (0,6) node[left] {$v$};

        % Shading the region R
        \fill[blue!20] (0.25,1) rectangle (2.5,5);

        % Lines $u = 1/4$ and $u = 5/2$
        \draw[dashed] (0.25,0) -- (0.25,6) node[above] {$u = \frac{1}{4}$};
        \draw[dashed] (2.5,0) -- (2.5,6) node[above] {$u = \frac{5}{2}$};
        \node at (1.375, 3) {$S$}; % Label S

        % Lines $v = 1$ and $v = 5$
        \draw[dashed] (0,1) -- (3,1) node[right] {$v = 1$};
        \draw[dashed] (0,5) -- (3,5) node[right] {$v = 5$};
    \end{tikzpicture}
    \caption{Graph of the region $S$}
\end{minipage}
\end{figure}

    \item Find the limits of integration for the new integral with respect to $u$ and $v$.
    \begin{empheq}[box=\fbox]{align*}
    \frac{1}{4} &\leq u \leq \frac{5}{2} \\
    1 &\leq v \leq 5
\end{empheq}
    \item Compute the Jacobian.
    $$J(u, v) = \begin{vmatrix}
  \dfrac{\partial u}{\partial x} & \dfrac{\partial v}{\partial x}\\[1em]
  \dfrac{\partial u}{\partial y} & \dfrac{\partial v}{\partial y} \end{vmatrix}
= \begin{vmatrix}
  \dfrac{-y}{u^2} & \dfrac{1}{y}    \\[1em]
  x     & \dfrac{1}{x} \end{vmatrix}= \left(\dfrac{-y}{u^2}\cdot \dfrac{1}{x}\right) - \left(\dfrac{x}{y}\right) =  \frac{-1}{2u}$$
    \item Change the variables to find the mass of lamina $R$ assuming that $R$ has constant density.
    \begin{align*}
        \int_{1}^{5}\int_{1/4}^{5/2} \delta\left| \frac{-1}{2u}\right| \, du \, dv &=\\
        &= \frac{1}{2}  \int_{1}^{5} \left[\ln(u)\right]_{1/4}^{5/2} \, dv\\
        &= \frac{1}{2}  \int_{1}^{5} \left[\ln\left(\frac{5}{2}\right)-\ln\left(\frac{1}{4}\right)\right] \, dv\\
        &= \frac{1}{2} \ln (10)  \cdot\int_{1}^{5}  \delta\, dv\\
        &= \frac{1}{2}\ln\left(10\right)  \cdot 4\cdot \delta\\
        &= 2\ln\left(10\right) \delta\\
    \end{align*}
\end{enumerate}

\section{Section 16.7}
Use the transformation $u = x + 2y$, $v = y-x$ to evaluate $\displaystyle \int_{0}^{\frac{2}{3}} \int_{y}^{2-2y}\left(x+2y\right)e^{y-x} \, dx \, dy$.


$$J(u,v) = \left|\begin{matrix} \dfrac{\partial u}{\partial x} & \dfrac{\partial u}{\partial y} \\[1em]\dfrac{\partial v}{\partial x} & \dfrac{\partial v}{\partial y} \end{matrix} \right| = \left| \begin{matrix}
\dfrac{1}{3} & \dfrac{1}{3}\\[1em]
-\dfrac{2}{3} & \dfrac{1}{3}
\end{matrix}\right| = \left(\frac{1}{3}\right)\left(\frac{1}{3}\right)-\left(-\frac{2}{3}\right)\left(\frac{1}{3}\right)=\frac{1}{3}$$

\begin{enumerate}
    \item $x = \frac{1}{3}(u - 2v)$
    \item $y = \frac{1}{3}(u + v)$
\end{enumerate}

Original limits of integration: 
$$0\leq y \leq \frac{2}{3}\,,\qquad y\leq x \leq 2-2y\,.$$

Bound of $y$:
\begin{align*}
0\leq y \leq \frac{2}{3}&\Longrightarrow 0\leq u+v \leq 2\,\\
\end{align*}

Bound of $x$:
\begin{align*}
y\leq x \leq 2-2y &\Longrightarrow u+v\leq u-2v\leq 6-2(u+v) &\Longrightarrow \begin{cases*}v\leq 0\\u\leq2\end{cases*} \\
\end{align*}

New limits of integration:
$$0\leq u\leq 2\,,\qquad-u\leq v\leq0\,.$$

Integration by parts:
\begin{enumerate}
    \item Let $w = u \Longrightarrow dw = du$
    \item Let $dt = e^{-u} \Longrightarrow t = -e^{-u}$
\end{enumerate}

\begin{align*}
    \int_0^2 \int_{-u}^{0} \left(u\cdot e^v\right) \cdot |J| \, dv \, du &= \\
    &= \frac{1}{3} \int_0^2 \int_{-u}^{0} \left(u\cdot e^v\right) \, dv \, du\\
    &= \frac{1}{3}\left[\frac{u^2}{2}-\left(-ue^{-u} -\int -e^{-u} \, du \right)\right]_{0}^{2}\\
    &= \frac{1}{3}\left[\frac{u^2}{2}+ue^{-u} + e^{-u}\right]_{0}^{2}\\
    &= \frac{1}{3}\left[\frac{u^2}{2}+\frac{1+u}{e^{u}}\right]_{0}^{2}\\
    &= \frac{1}{3}\left(\left(2+\frac{3}{e^{2}}\right)-\left(1\right)\right)\\
    &= \frac{2}{3}+\frac{1}{e^{2}}-\frac{1}{3}\\
    &= \boxed{\frac{1}{3}+\frac{1}{e^{2}}}
\end{align*}


\end{document}
