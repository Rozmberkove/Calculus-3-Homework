\documentclass[letter,11pt]{article}
\usepackage{latexsym}
\usepackage{xcolor}
\usepackage{float}
\usepackage{amsthm}
\usepackage{amssymb}
\usepackage{wrapfig}
\usepackage{tabularx}
\usepackage{titlesec}
\usepackage{tikz}
\usepackage{geometry}
\usepackage{verbatim}
\usepackage{enumitem}
\usepackage{fancyhdr}
\usepackage{pgfornament}
\usepackage{multicol}
\usepackage{systeme}
\usepackage{graphicx}
\usepackage{mathtools}
%\usepackage{cfr-lm}
\usepackage{booktabs}
\usepackage{svg}
\usepackage[T1]{fontenc}
\usetikzlibrary{trees}
\setlength{\multicolsep}{0pt} 
\pagestyle{fancy}
%\fancyhf{} % clear all header and footer fields
\fancyhead{}\fancyfoot{}
\fancyhead[R]{\textbf{\thepage}}
\fancyhead[L]{Aiden M. Rosenberg, MMXXIII A.D. }
\addtolength{\headwidth}{3cm}

\usepackage{pgfplots}
\pgfplotsset{compat=1.17}


\renewcommand{\headrulewidth}{1pt}
\renewcommand{\footrulewidth}{0pt}
\geometry{left=1.5cm, top=2.5cm, right=1.5cm, bottom=2cm}

%\usepackage{draftwatermark}	
%\SetWatermarkColor[gray]{0.9}
%\SetWatermarkText{Private}
%\SetWatermarkScale{3}

\usepackage[most]{tcolorbox}
\tcbset{
	frame code={}
	center title,
	left=0pt,
	right=0pt,
	top=0pt,
	bottom=0pt,
	colback=gray!20,
	colframe=white,
	width=\dimexpr\textwidth\relax,
	enlarge left by=-2mm,
	boxsep=4pt,
	arc=0pt,outer arc=0pt,
}


\raggedright
\setlength{\tabcolsep}{0in}

% Sections formatting
\titleformat{\section}{
  \vspace{-4pt}\scshape\raggedright\large
}{}{0em}{}[\color{black}\titlerule \vspace{-7pt}]

\begin{document}

\thispagestyle{empty}

\fontfamily{cmr}\selectfont
%----------HEADING-----------------

\parbox{2.35cm}{%
  \includesvg[width=2.3cm]{logo.svg}
}
\parbox{0.3cm}{\hspace{0.3cm}}
\parbox{\dimexpr\linewidth-5cm\relax}{
\setlength{\tabcolsep}{0.5em}
\def\arraystretch{1.25}
\begin{tabular}{@{}llll@{}}
\toprule
 \multicolumn{4}{c}
 {\hspace{-0.5em}\textbf{Assignment}: Worksheet \#5 (15.8)} \\ \midrule
\textbf{Name:}      & D. Aiden M. Rosenberg    & \textbf{Professor:}   & Dr. Alan v. Herrmann Ph.D        \\
\textbf{Course:}    & Calculus III        & \textbf{Date:}        & September 26th, 2023 A.D.   \\ \bottomrule
\end{tabular}
}
\vspace{1cm}

\section*{Section 15.8.1}
The function $f (x, y) = 4xy$ has an absolute maximum and absolute minimum subject to constraint $4x^2 + y^2 \leq 8$. Use Lagrange multipliers to find these values.
\begin{enumerate}[label=\roman*.]
    \item $f(x,y)=4xy \Longrightarrow \nabla f(x,y) = \langle 4y,4x\rangle$
    \item $g(x,y) = 4x^2+y^2 \Longrightarrow \langle 8x,2y\rangle$
\end{enumerate}
\[
\begin{cases}
    4y = 8x\lambda\\
    4x = 2y\lambda\\
    \;\; 1 = x^2 + y^2\\
\end{cases}
\]

$$\Longrightarrow \frac{4y}{4x} = \frac{8x\lambda}{2y\lambda} = \frac{4x}{y} \Longrightarrow y^2 = 3x^2 \Longrightarrow y=2x$$
\begin{align*}
    8 &= 4x^2 +(2x)^2 = 8\\
    &= 8x^2
\end{align*}
$$\Longrightarrow x=\pm 1 \Longrightarrow y=\pm 1$$
The relative extreme are:
\begin{enumerate}
    \item $f(1,2) = 8$
    \item $f(-1,-2) = 8$
    \item $f(1,-2)= -8$
    \item $f(-1,2) = -8$
\end{enumerate}
\fbox{\parbox{\textwidth}{There is an absolute maximum $8$ at the points $(1,2)$ \& $(-1,-2)$, and absolute minimum of $-8$ at the points $(-1,2)$ and $(1,-2)$}}


\section*{Section 15.8.2}
Use Lagrange multipliers to find the point on $x^2 + y^2 + z^2 = 20$ that is farthest from $P (-2, 0, 1)$. Determine this maximal distance.
\begin{enumerate}[label=\roman*.]
    \item Let $g(x,y,z) = x^2 + y^2 + z^2 = 20 \Longrightarrow \nabla g \langle 2x,2y,2z\rangle$
    \item $d = \sqrt{(x+2)^2+(y-0)^2+(z-1)^2} = \sqrt{x^2+4x+y^2+z^2-2z+5}$
    \item Let $f(x,y,z) = x^2+4x+y^2+z^2-2z+5 \Longrightarrow \nabla f = \langle 2x+4,2y,2z-2\rangle$
\end{enumerate}

\[
\begin{cases}
    2x=\lambda(2x+4) \Longrightarrow x = \frac{4\lambda}{2-2\lambda}\\
    2y = 2y\lambda \Longrightarrow y = 0\\
    2z = \lambda(2z-2)\Longrightarrow z = \frac{-2\lambda}{2-2\lambda}
\end{cases}
\]

\begin{align*}
    20 &= \underbrace{\left(\frac{-2}{1-\lambda}\right)^2}_{x^2} + \underbrace{(0)^2}_{y^2} +\underbrace{\left(\frac{-1}{\lambda-1}\right)^2}_{z^2}\\
    &= \frac{5}{(\lambda-1)^2}
\end{align*}

$$\frac{5}{20} = (\lambda-1)^2 \Longrightarrow \lambda =  \frac{1}{2}, \frac{3}{2}$$

$$\lambda =  \frac{1}{2} \Longrightarrow x = 2 \text{ and } y= -4$$
$$\lambda =  \frac{3}{2} \Longrightarrow x = -2 \text{ and } y= 4$$


The relative extreme are:
\begin{enumerate}
    \item $f(4,0,-2)= 45$
    \item $f(-4,0,2) = 5$
\end{enumerate}
\fbox{\parbox{\textwidth}{The maximum distance from the point $P$ is $\sqrt{45} = 3\sqrt{5}$ and occurs at the point $(4,0,-2)$}}

\section*{Section 15.8.3}
A rectangular box (without a top) is to be constructed having a volume of 16 m$^3$ using two different materials. The material for the bottom is four times as costly (per square inch) as the rest of the material. Determine the dimensions of the box that will minimize the cost of materials using Lagrange multipliers.

\begin{enumerate}[label=\roman*.]
    \item $g(x,y,z) = xyz=16 \Longrightarrow \nabla g(x,y,z)= \langle yz, xz,xy\rangle$
    \item $f(x,y,z) = 4xy+2(xz+yz) = 4xy+2xz+2yz \Longrightarrow \nabla f(x,y,z) = \langle 4y+2z, 4x+2z, 2x+2y\rangle$
\end{enumerate}

$$
\begin{cases}
4y+2z = yz\lambda\\
4x+2z = xz\lambda \\
2x+2y = xy\lambda \\
xyz = 16
\end{cases}
\xRightarrow[]{\text{Multiply by missing variable}}
\begin{cases}
16\lambda = 4yx+2xz\\
16\lambda = 4xy+2yz\\
16\lambda = 2xz+2yz\\
xyz = 16
\end{cases}
$$
$$\Longrightarrow 0 = 4xy-2xz \Longrightarrow z=2y$$
$$\Longrightarrow 0 = 2xz-2yz \Longrightarrow y =x$$
$$\Longrightarrow 16=(y)(y)(2y) \Longrightarrow y = 2, x=2 \text{ and } z = 4$$
$$(2,2,4)$$
Dimensions: $\boxed{W= 2, L =2 \text{ and } H= 4}$



\section*{Integration Review}
Solve the following integrals: 
\begin{enumerate}[label=\roman*.]
    \item $\int x^2 e^{x^3} \, dx$
        \begin{enumerate}
            \item Let $u=x^3 \Longrightarrow du = 3x^2 \, dx$
        \end{enumerate}
       \begin{align*}
           \int x^2 e^{x^3} \, dx &= \\
           &= \frac{1}{3}\int e^{u} \, du \\
           &= \frac{1}{3}e^u + C\\
           &= \boxed{\frac{1}{3}e^{x^3} + C}
       \end{align*}
    \item $\int x^2 e^{-3x} \, dx$
        \begin{enumerate}
            \item Let $u=x^2 \Longrightarrow du = 2x \, dx$ and $dv = e^{-3x} \,dx \Longrightarrow v = \frac{-e^{-3x}}{3}$
            \item Let $s=x \Longrightarrow dx =ds$ and $dt = e^{-3t} \, dx \Longrightarrow \frac{-1}{3}e^{-3x}$
        \end{enumerate}
        \begin{align*}
            \int x^2 e^{-3x} \, dx & = \\
           &= \frac{-x^2e^{-3x}}{3} + \frac{2}{3} \int xe^{-3x}\, dx \\ 
           &= \frac{-x^2e^{-3x}}{3} + \frac{2}{3} \left[\frac{-1}{3}xe^{-3x} -\int \frac{-1}{3} e^{-3x} \, dx \right] \\
           &= \frac{-x^2e^{-3x}}{3} + \frac{2}{3} \left[\frac{-1}{3}xe^{-3x} - \frac{1}{9} e^{-3x} \right]\\
           & = \frac{-x^2e^{-3x}}{3} - \frac{2xe^{-3x}}{9}-\frac{2e^{-3x}}{27}\\
           &= \boxed{\frac{-(9x^2+6x+2)e^{-3x}}{27}+C}
        \end{align*}
    \item $\int_{1}^{e} \frac{\ln x}{x^2} \, dx$
        \begin{enumerate}
            \item Let $u=\ln x \Longrightarrow du = \frac{1}{x} \, dx$
        \end{enumerate}
        \begin{align*}
            \int_{1}^{e} \frac{\ln x}{x^2} \, dx &=\\
            &= \left[\frac{\ln x}{x}\right]_{0}^{1} + \int_{0}^{1}  \frac{-1}{x^2} \, dx \\
            &= \left[\frac{\ln x}{x} - \frac{1}{x}\right]_{1}^{e} \\ 
            &= \left(\frac{-1}{e}-\frac{1}{e}\right) - (0-1)\\
            &= \boxed{1-2e^{-1}}
        \end{align*}
        
    \item $\int_{0}^{\pi/6} \sin (2\theta)\cos^3(2\theta) \, d\theta$
        \begin{enumerate}
            \item Let $u = \cos(2\theta) \Longrightarrow -2\sin(2\theta) \, d\theta$
        \end{enumerate}
    \begin{align*}
        \int_{0}^{\pi/6} \sin (2\theta)\cos^3(2\theta) \, d\theta \\
        &= \frac{1}{2} \int_{1/2}^{1} u^3 \, dt \\ 
        &= \frac{1}{2} \cdot\left(\frac{u^4}{4}\right) = \frac{1}{8}u^4\\
        &= \frac{1}{8}u^4 \\ 
        &= \frac{1}{8}\left(\frac{1}{16}+ 1\right)\\
        &= \frac{1}{8} \cdot \frac{15}{8} \\
        &= \boxed{\frac{15}{128}}
    \end{align*}
    
    \item $\int_{0}^{\pi/4} \cos^2(2\theta)\, d\theta$
    \begin{enumerate}
        \item Let $u =2\theta \Longrightarrow du = d\theta$
    \end{enumerate}
        \begin{align*}
            \int_{0}^{\pi/4} \cos^2(2\theta)\, d\theta &=\\ 
            &= \frac{1}{2} \int_{0}^{\pi/2} \cos^2(u) \, du\\
            &= \frac{1}{2} \int_{0}^{\pi/2} \frac{(1+\cos(2x)}{2} \, d\theta\\
            &= \frac{1}{2} \left[\frac{1}{2}\theta + \frac{1}{4}\sin(2\theta)\right]_{0}^{\pi/2}\\
            &= \boxed{\frac{\pi}{8}}
        \end{align*}
    \item $\int_{0}^{\pi/3} \cos^3(2\theta)\, d\theta$
    \begin{enumerate}
        \item Let $u=2\theta \Longrightarrow du = 2 \, d\theta$
        \item Let $v = \sin u \Longrightarrow dv = \cos u \, du$
    \end{enumerate}
    \begin{align*}
        \int_{0}^{\pi/3} \cos^2(2\theta)\, d\theta & = \\
        &= \frac{1}{2} \int_{0}^{2\pi/3} \cos^3 u \, du\\
         &= \frac{1}{2} \int_{0}^{2\pi/3} \left((1-\sin^2(u))\cdot \cos(u)\right) \, du\\
         &= \frac{1}{2} \int_{0}^{2\pi/3} \left(\cos x - \cos x\sin^2 x \right) \, du\\
         &= \frac{1}{2} \left[\left(\sin \left(\frac{2\pi}{3}\right)- \sin \left(0\right) \right) - \int_{0}^{\frac{\sqrt{3}}{2}} u^2 \, du\right]\\
         &= \frac{-1}{2} \left[\frac{\sqrt{3}}{4}-\frac{u^3}{3}\right]_{0}^{\frac{\sqrt{3}}{2}}\\
         &= \boxed{\frac{3\sqrt{3}}{16}}
    \end{align*}
\end{enumerate}
\end{document}
