\documentclass[letter,11pt]{article}
\usepackage{latexsym}
\usepackage{xcolor}
\usepackage{float}
\usepackage{amsthm}
\usepackage{esint}
\usepackage{amssymb}
\usepackage{wrapfig}
\usepackage{tabularx}
\usepackage{empheq}
\usepackage{titlesec}
\usepackage{tikz}
\usepackage{geometry}
\usepackage{verbatim}
\usepackage{epstopdf}
\usepackage{enumitem}
\usepackage{fancyhdr}
\usepackage{pgfornament}
\usepackage{multicol}
\usepackage{systeme}
\usepackage{graphicx}
\usepackage{mathtools}
%\usepackage{cfr-lm}
\usepackage{booktabs}
\usepackage{svg}
\usepackage[T1]{fontenc}
\usetikzlibrary{trees}
\setlength{\multicolsep}{0pt} 
\pagestyle{fancy}
%\fancyhf{} % clear all header and footer fields
\fancyhead{}\fancyfoot{}
\fancyhead[R]{\textbf{\thepage}}
\fancyhead[L]{Aiden M. Rosenberg, MMXXIII A.D. }
\addtolength{\headwidth}{3cm}

\usepackage{pgfplots}
\pgfplotsset{compat=1.17}
\usepgfplotslibrary{fillbetween}

\usepackage{pst-plot}
\usepgfplotslibrary{polar}



\renewcommand{\headrulewidth}{1pt}
\renewcommand{\footrulewidth}{0pt}
\geometry{left=1.5cm, top=2.5cm, right=1.5cm, bottom=2cm}

%\usepackage{draftwatermark}	
%\SetWatermarkColor[gray]{0.9}
%\SetWatermarkText{Private}
%\SetWatermarkScale{3}

\usepackage[most]{tcolorbox}
\tcbset{
	frame code={}
	center title,
	left=0pt,
	right=0pt,
	top=0pt,
	bottom=0pt,
	colback=gray!20,
	colframe=white,
	width=\dimexpr\textwidth\relax,
	enlarge left by=-2mm,
	boxsep=4pt,
	arc=0pt,outer arc=0pt,
}


\raggedright
\setlength{\tabcolsep}{0in}

% Sections formatting
\titleformat{\section}{
  \vspace{-4pt}\scshape\raggedright\large
}{}{0em}{}[\color{black}\titlerule \vspace{-7pt}]

\titleformat{\subsection}[block]
  { \vspace{4pt}\bfseries\centering}
  {}{0em}{}

\newcommand{\pvec}[1]{\vec{#1}\mkern2mu\vphantom{#1}}

\begin{document}

\thispagestyle{empty}

\fontfamily{cmr}\selectfont
%----------HEADING-----------------

\parbox{2.35cm}{%
	\includesvg[width=2.3cm]{logo.svg}
}
\parbox{0.3cm}{\hspace{0.3cm}}
\parbox{\dimexpr\linewidth-5cm\relax}{
	\setlength{\tabcolsep}{0.5em}
	\def\arraystretch{1.25}
	\begin{tabular}{@{}llll@{}}
		\toprule
		\multicolumn{4}{c}
		{\hspace{-0.5em}\textbf{Assignment}: Worksheet \#12 (17.6)} \\ \midrule
		\textbf{Name:}   & D. Aiden M. Rosenberg & \textbf{Professor:} & Dr. Alan v. Herrmann Ph.D \\
		\textbf{Course:} & Calculus III          & \textbf{Date:}      & \today \: A.D.   \\ \bottomrule
	\end{tabular} }
\vspace{1cm}
\section*{Section 17.6.1}
Find the area of the part of $x^{2}+y^{2}=9$ between $z=0$ and $z=3+y$.
\begin{enumerate}[label = \roman*.]
    \item Let $\vec{r}\left(u,v\right) = \langle 3\cos u, 3\sin u, v\rangle$
    \item $z=3+y \Longrightarrow v=3+3\sin u$
\end{enumerate}
$$ r_{u}\times r_{v}=
\begin{vmatrix}
\hat{i} & \hat{j} & \hat{k}\\[1em]
\dfrac{\partial x}{\partial u} & \dfrac{\partial y}{\partial u} & \dfrac{\partial z}{\partial u}\\[1em]
\dfrac{\partial x}{\partial v} & \dfrac{\partial v}{\partial v} & \dfrac{\partial z}{\partial v}
\end{vmatrix}
=
\begin{vmatrix}
\hat{i} & \hat{j} & \hat{k}\\
-3\sin u & 3\cos u & 0\\
0 & 0 & 1
\end{vmatrix}
= \langle 3\cos u, 3\sin u, 0 \rangle
$$
$$\left|r_{u}\times r_{v}\right| = \sqrt{\left(3\cos u\right)^2+\left(3\sin u\right)^2} = \sqrt{9\left(\sin^2 u +\sin^2 u\right)}=3$$

\begin{align*}
    A\left(S\right)=\iint_{D}\left|r_{u}\times r_{v}\right| \, dA &= \\
    &= \int_{0}^{2\pi} \int_{0}^{3+3\sin u} 3 \, dv \, du\\
    &= \int_{0}^{2\pi} \left(9+9\sin u\right) \, du\\
    &= \left[9u-9\cos u\right]_{0}^{2\pi}\\
    &= \left(18\pi-1\right)-\left(0-1\right)\\
    &= \boxed{18\pi}
\end{align*}

\section*{Section 17.6.2}
Evaluate $\iint_{S} y^{2} d S$ where $S$ is the part of $z=3+y$ inside $x^{2}+y^{2}=9$.
\begin{enumerate}[label = \roman*.]
    \item Let $\vec{r}\left(\theta, r\right) = \langle r\cos \theta, r\sin \theta\rangle$ where $0\leq r \leq 3$ and $0 \leq \theta \leq 2\pi$
    \item $\frac{\partial z}{\partial x} = 0$
    \item $\frac{\partial z}{\partial y} = 1$
\end{enumerate}
\begin{align*}
    \iint_{D} y^2 \sqrt{\left(\dfrac{\partial z}{\partial x}\right)^2+\left(\dfrac{\partial z}{\partial y}\right)^2 +1} \, dA &=\\
    &= \sqrt{2}\int_{0}^{2\pi}\int_{0}^{3}\left(r\sin\left(\theta\right)\right)^{2}r\ dr\ d\theta \\
    &= \sqrt{2}\int_{0}^{2\pi}\int_{0}^{3}r^{3}\sin^{2}\left(\theta\right) \, dr\, d\theta \\
    &= \frac{81\sqrt{2}}{4}\int_{0}^{2\pi}\sin^{2}\left(\theta\right) \,d\theta\\
    &= \frac{81\sqrt{2}}{8}\int_{0}^{2\pi}\left(1-\sin\left(2\theta\right)\right)\, d\theta\\
    &= \frac{81\sqrt{2}}{8} \left[\theta-\frac{\cos\left(2\theta\right)}{2}\right]_{0}^{2\pi}\\
    &= \frac{81\sqrt{2}}{8} \left(2\pi\right)\\
    &= \boxed{\frac{81\pi\sqrt{2}}{4}}
\end{align*}

\section*{Section 17.6.3} 
Determine the upward flux of $\vec{F}=\langle-y z, x z, z^{2}\rangle$ through the part of $z=\sqrt{x^{2}+y^{2}}$ below $z=2$.
\begin{enumerate}[label = \roman*.]
    \item $\vec{r}\left(\theta,r\right)=\langle r\cos \theta ,r\sin \theta,r\rangle$
    \item $\vec{F}=\langle-y z, x z, z^{2}\rangle = \langle -r^{2}\sin \theta,r^{2}\cos \theta,r^{2}\rangle$
\end{enumerate}
$$ r_{u}\times r_{v}=
\begin{vmatrix}
\hat{i} & \hat{j} & \hat{k}\\[1em]
\dfrac{\partial x}{\partial \theta} & \dfrac{\partial y}{\partial \theta} & \dfrac{\partial z}{\partial \theta}\\[1em]
\dfrac{\partial x}{\partial r} & \dfrac{\partial y}{\partial r} & \dfrac{\partial z}{\partial r}
\end{vmatrix}
=
\begin{vmatrix}
\hat{i} & \hat{j} & \hat{k}\\
-r\sin \theta & r\cos \theta & 0\\
\cos \theta & \sin \theta & 1
\end{vmatrix}
= \langle r\cos \theta, r\sin \theta, -r \rangle
$$
\begin{align*}
\iint_{D} \vec{F}\left(\vec{r}\left(\theta,r\right)\right) \cdot \left(\vec{r}_{\theta}\times \vec{r}_{r}\right) \, dA &=\\
    &= \int_{0}^{2\pi}\int_{0}^{2} \langle -r^{2}\sin \theta,r^{2}\cos u,r^{2}\rangle \cdot \langle -r\cos u, -r\sin u, r \rangle \, dr \, d\theta\\
    &=  \int_{0}^{2\pi}\int_{0}^{2} \left(r^3\cos \theta \sin \theta-r^3\sin \theta \cos \theta +r^3\right) \, dr \, d\theta\\
    &= \int_{0}^{2\pi}\int_{0}^{2}r^{3}\left(1\right) \, dr \, d\theta\\
    &= 4\int_{0}^{2\pi}\left(1\right)\, d\theta\\
    &= \boxed{8\pi}
\end{align*}
\end{document}
